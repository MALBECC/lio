\documentclass[journal=jctcce,manuscript=article]{achemso}
%journal=ancac3, % for ACS Nano
%journal=acbcct, % for ACS Chem. Biol.
%journal=jacsat, % for undefined journal
%journal=jctcce, % for ACS JTCT
%manuscript=article]{achemso}
\setkeys{acs}{articletitle = true}
%\documentclass[aps,floats,prb,12pt]{revtex4} ESTE
\usepackage{graphics,dcolumn}
\usepackage{graphicx}
\usepackage{chemfig}
\usepackage[version=3]{mhchem}
\usepackage{float}
\usepackage{comment}
\usepackage{soul}
\usepackage[flushleft]{threeparttable}
\usepackage{multirow}
\renewcommand{\baselinestretch}{2}
%\bibliographystyle{unsrt}

%\begin{document}
\special{papersize=8.5in,11in}

\title{Lio}
%======================================================================

\author{NaN}
\affiliation{Departamento de Qu\'imica Inorg\'anica, Anal\'itica
y Qu\'imica F\'isica/INQUIMAE, Facultad de Ciencias Exactas
y Naturales, Universidad de Buenos Aires, Ciudad Universitaria,
Pab. II, Buenos Aires (C1428EHA) Argentina}


\begin{document}


\begin{abstract}
proyecto de manual de lio, hay q ponerle el formato adecuado
\end{abstract}

%\date{\today}
%\pacs{}
%\maketitle

\newpage
\section{Quick Reference}
This section provides a quick reference for LIO input variables, also providing the default values.

    \subsection{File I/O}
    \begin{table}  [H]
      \begin{center}
      \begin{tabular}{ l c c l}
         Variable & Type & Default & Description \\
         basis    & char*20 & 'basis' & Filename for the basis set file (if a custom basis set is used).   \\
         output & char*20 & 'output' & Output file. \\
         fCoord & char*20 & 'qm.xyz' & xyz file (without the header) containing the QM System \\
            & & & coordinates. \\
         writexyz & logical & false & Writes an xyz file containing the QM system coordinates, \\
            & & & useful when using LIO in tandem with AMBER/GROMACS. \\
         verbose & integer & 1 & Verbose level. \\
         VCInp & logical & false & \\
         restart\_freq & integer & 1 & \\
         frestart & char*20 & 'restart.out' & Output restart file. \\
         frestartin & char*20 & 'restart.in' & Input restart file. \\
       \end{tabular}
       \end{center}
      \label{lio.fileio.var}
    \end{table}

    \subsection{Electronic Properties}
    \begin{table}  [H]
      \begin{center}
      \begin{tabular}{ l c c l}
         Variable & Type & Default & Description \\
         writeDens & logical & false & Writes electronic density to an output file after calculation. \\
         writeForces & logical & false & Writes final forces to output. \\
         print\_coeffs & logical & false & Prints MO coefficients in AO basis. \\
         mulliken & logical & false & Performs a Mulliken Population Analysis. \\
         lowdin & logical & false & Performs a Lowdin Population Analysis. \\
         fukui & logical & false & Calculates condensed-to-atoms Fukui function (Spin Polarized \\
            & & & Fukui in open-shell systems).
       \end{tabular}
       \end{center}
      \label{lio.properties.var}
    \end{table}

    \subsection{Theory Level Options}
    \begin{table}  [H]
      \begin{center}
      \begin{tabular}{ l c c l}
         Variable & Type & Default & Description \\
         natom & integer & 0 & Number of QM atoms (ignored from \\
            & & & AMBER/GROMACS). \\
         nsol & integer & 0 & Number of classical atoms (ignored from \\
            & & & AMBER/GROMACS). \\
         charge & integer & 0 & QM system total charge. \\
         open & logical & false & Perform an open-shell calculation. \\
         nUnp & integer & 0 & Number of unpaired electrons. \\
         int\_basis & logical & true & If set to false, an external basis file must be \\ 
            & & & provided. \\
         basis\_set & char*20 & 'DZVP' & Name of the basis set used in the calculation. \\
            & & & Ignored if int\_basis is set to false. \\
         fitting\_set & char*40 & 'DZVP Coulomb Fitting' & Name of the fitting set used in the calculation. \\
            & & & Ignored if int\_basis is set to false. \\
         nMax & integer & 100 & Maximum number of SCF steps. \\
         number\_restr & integer & 0 & Amount of distance restraints used. \\
      \end{tabular}
       \end{center}
      \label{lio.theory.var}
    \end{table}  
    
    \subsection{Theory Level Advanced Options}
    \begin{table}  [H]
      \begin{center}
      \begin{tabular}{ l c c l}
         Variable & Type & Default & Description \\
         DIIS & logical & true & Use DIIS convergence. \\
         nDIIS & integer & 30 & \\
         hybrid\_converg & logical & false & Use Hybrid convergence. \\
         gold & real*8 & 10 & \\
         told & real*8 & 1.0D-6 & \\
         ETold & real*8 & 1.0D0 & \\
         good\_cut & real*8 & 1.0D-5 & \\
         rmax & real*8 & 16 & \\
         rmaxs & real*8 & 5 & \\
         omit\_bas & logical & false & \\
         predCoef & logical & false & \\
         dgTrig & integer & 100 & \\
         iExch & integer & 9 & \\
         integ & logical & true & \\
         intSolDouble & logical & true & \\
         dens & logical & true & \\
         iGrid & integer & 2 & \\
         iGrid2 & integer & 2 & \\         
       \end{tabular}
       \end{center}
      \label{lio.theorya.var}
    \end{table}    
    
    \subsection{TD-DFT Options}
    \begin{table}  [H]
      \begin{center}
      \begin{tabular}{ l c c l}
         Variable & Type & Default & Description \\
         timeDep & integer & 0 & Use TDDFT (when =1). \\
         tdStep & real*8 & 2.D-5 & Timestep for TD-DFT (in atomic units). \\
         ntdStep & integer & 0 & Maximum number of TD-DFT steps. \\
         propagator & integer & 1 & \\
         NBCH & integer & 10 & \\
         field & logical & false & \\
         epsilon & real*8 & 1.D0 & \\
         a0 & real*8 & 1000 & \\
         exter & logical & false & Apply an external electric field. \\
         Fx, Fy, Fz & real*8 & 0.05 &  The strength of the external electric field. \\
         tdrestart & logical & false & \\
       \end{tabular}
       \end{center}
      \label{lio.tddft.var}
    \end{table}    

    \subsection{Effective Core Potential Options}
    \begin{table}  [H]
      \begin{center}
      \begin{tabular}{ l c c l}
         Variable & Type & Default & Description \\
         ECPMode & logical & false & Activate the ECP mode. \\
         ECPTypes & integer & 0 & \\
         tipeECP & char*20 & 'NOT-DEFINED' & \\
         ZListECP & integer & 0 & \\
         cutECP & logical & true & \\
         ECP\_debug &  logical & false & \\
         local\_nonlocal & integer & 0 & \\
         ECP\_full\_range\_int & logical & false & \\
         verbose\_ECP & integer & 0 & \\
         fock\_ECP\_read & logical & false & \\
         fock\_ECP\_write & logical & false & \\
         fullTimer\_ECP & logical & false & \\
         cut2\_0 & real*8 & 15.D0 & \\
         cut3\_0 & real*8 & 12.D0 & \\
       \end{tabular}
       \end{center}
      \label{lio.ecp.var}
    \end{table}    

    \subsection{Orbital Printing Options}
    \begin{table}  [H]
      \begin{center}
      \begin{tabular}{ l c c l}
         Variable & Type & Default & Description \\
         cubeGen\_only & logical & false & \\
         cube\_res & integer & 40 & \\
         cube\_sel & integer & 0 & \\
         cube\_dens & logical & false & Prints the electronic density.  \\
         cube\_dens\_file & char*20 & 'dens.cube' & File containing the electronic density. \\
         cube\_orb & logical & false & Prints orbital shapes. \\
         cube\_orb\_file & char*20 & 'orb.cube' & File containing the orbital shapes. \\
         cube\_elec & logical & false & Prints the electric field. \\
         cube\_elec\_file & char*20 & 'field.cube' & File containing the electric field. \\
         cube\_sqrt\_orb & logical & false & \\
       \end{tabular}
       \end{center}
      \label{lio.ecp.var}
    \end{table}    
    
    \subsection{GPU Options}
    \begin{table}  [H]
      \begin{center}
      \begin{tabular}{ l c c l}
         Variable & Type & Default & Description \\
         max\_function\_exponent & integer & 10 & \\
         little\_cube\_size & real*8 & 8.0 & \\
         min\_points\_per\_cube & integer & 1 & \\
         assign\_all\_functions & logical & false & \\
         sphere\_radius & real*8 & 0.6 & \\
         remove\_zero\_weights & logical & true & \\
         energy\_all\_iterations & logical & false & \\
         free\_global\_memory & real*8 & 0.0 & \\
       \end{tabular}
       \end{center}
      \label{lio.ecp.var}
    \end{table}    

\newpage
\section{Restraints}
LIO may add an extra potential term to the Hamiltonian in order to restrain the distance between specified pairs of atoms.

    \subsection{Implemenation}
    The implementation is a simple harmonic potential over a generalized coordinate $r$.

    \begin{equation}
      U=\frac{1}{2} k [r - l_0]^2  
      \label{E_restrain}
    \end{equation}

    $r$ may be defined as a weighted combination of distances between pairs of atoms.

    \begin{equation}
      r=  \sum_{i} \sum_{j>i} w_{ij} |\vec{r_i} - \vec{r_j}|
      \label{gen_coord}
    \end{equation}

    In this formulation the force over an atom l is:

    \begin{equation}
      \vec{F_l}= -k [r - l_0] \sum_{i} \sum_{j>i} w_{ij} \frac{\vec{r_{ij}}}{r_{ij}} \eta_{ijl}     
      \label{rest_force}
    \end{equation}

    Where $\eta_{ijl}$ is defined as:

    \begin{equation*}
      \eta_{ijl} =
       \begin{cases}
         1 & \text{if $l=i$}\\
         -1 & \text{if $l=j$}\\
         0 & \text{in other case}
       \end{cases}
       \label{eta}
    \end{equation*}


    \subsection{Using Restraints}

    The number of pairs of atoms to be added in the restraint potential(s) is defined by setting the variable number\_restr, and a list of distance restrains have to be added to in an additional lio.restrain file. For example:

    \begin{table}  [H]
      \begin{center}
      \begin{tabular}{ l c c c c c}
         $a_i$ & $a_j$ & index &   k  &    $w_{ij}$   &  $l_0$    \\
         1  &  2 &   0   &  0.1 &    1.0   & 7.86   \\
         3  &  4 &   0   &  0.1 &   -1.0   & 7.86   \\
         7  &  9 &   1   &  0.4 &    2.0   & -2.3   \\
         13 &  1 &   1   &  0.4 &    1.0   & -2.3   \\
         14 &  3 &   1   &  0.4 &   -3.0   & -2.3   \\
         14 &  2 &   2   &  0.2 &    1.0   & 0.5    \\
         8  &  5 &   3   &  0.3 &    1.0   & 3.2    \\
       \end{tabular}
       \end{center}
      \label{lio.restrain}
    \end{table}

Columns $a_i$ and $a_j$ contain the atom numbers in the QM system to be restrained, while the index number determines which distances contribute to a same generalized reaction coordinate. The remaining columns are the force constants (k), weights of that distance in the generalized coordinate ($w_{ij}$) and equilibrium positions in atomic units ($l_0$).

    \subsection{Examples}

    \textbf{1)In lio.in:}
    
        \textbf{in lio.restrain:}

    \begin{table}  [H]
      \begin{center}
      \begin{tabular}{ l c c c c c}
         $a_i$ & $a_j$ & index &   k  &    $w_{ij}$   &  $l_0$   \\
         1  &  2 &   0   &  0.1 &    1.0   & 7.86   \\
       \end{tabular}
       \end{center}
      \label{Tex1}
    \end{table}

    \textbf{Potential added to system:}

    \begin{equation}
      U=\frac{1}{2} 0.1 \Big{[} 1.0 |\vec{r_1} - \vec{r_2}| - 7.86\Big{]}^2  
      \label{Ex1}
    \end{equation}


    \textbf{2)In lio.in:}

    number\_restr = 2

    \textbf{in lio.restrain:}

    \begin{table}  [H]
      \begin{center}
      \begin{tabular}{ l c c c c c}
         $a_i$ & $a_j$ & index &   k  &    $w_{ij}$   &  $l_0$    \\
         1  &  2 &   0   &  0.1 &    1.0   & 7.86   \\
         3  &  4 &   0   &  0.1 &   -1.0   & 7.86   \\
       \end{tabular}
       \end{center}
      \label{Tex2}
    \end{table}

    \textbf{Potential added to system:}

    \begin{equation}
      U=\frac{1}{2} 0.1 \Big{[} 1.0 |\vec{r_1} - \vec{r_2}| - 1.0 |\vec{r_3} - \vec{r_4}| - 7.86\Big{]}^2  
      \label{Ex2}
    \end{equation}


    \textbf{3)In lio.in:}

    number\_restr = 4

    \textbf{in lio.restrain:}

    \begin{table}  [H]
      \begin{center}
      \begin{tabular}{ l c c c c c}
         $a_i$ & $a_j$ & index &   k  &    $w_{ij}$   &  $l_0$    \\
         1  &  2 &   0   &  0.1 &    1.0   & 7.86   \\
         3  &  4 &   0   &  0.1 &   -1.0   & 7.86   \\
         1  &  3 &   1   &  0.3 &    3.5   & -2.31   \\
         7  &  8 &   1   &  0.3 &   -2.2   & -2.31   \\
       \end{tabular}
       \end{center}
      \label{Tex3}
    \end{table}

    \textbf{Potential added to system:}

    \begin{equation}
      U=\frac{1}{2} 0.1 \Big{[} 1.0 |\vec{r_1} - \vec{r_2}| - 1.0 |\vec{r_3} - \vec{r_4}| - 7.86\Big{]}^2 + \frac{1}{2} 0.3 \Big{[} 3.5 |\vec{r_1} - \vec{r_3}| - 2.2 |\vec{r_7} - \vec{r_8}| +2.31\Big{]}^2 
      \label{Ex3}
    \end{equation}



\end{document}
