%%%%%%%%%%%%%%%%%%%%%%%%%%%%%%%%%%%%%%%%%%%%%%%%%%%%%%%%%%%%
\chapter{Introduction}
%%%%%%%%%%%%%%%%%%%%%%%%%%%%%%%%%%%%%%%%%%%%%%%%%%%%%%%%%%%%
\section{What is LIO?}

Welcome to the LIO project! LIO is a library that can
perform electronic structure calculations using density
functional theory. 


\section{Instalation}

If you are reading this manual, you probably already have
a version of lio ready to compile. If you don't, or if
you want to make sure you have the most up-to-date version
of the code, all you need to do is either download it
from the git repository online or use git to clone a copy.

For the first option, go to
\textbf{\textit{\href
{https://github.com/MALBECC/lio}
{https://github.com/MALBECC/lio}
}}
and click on the green button that says 
\textit{clone or download}
and click on \textit{Download ZIP}.

For the second one, you can directly run the following
command:

\lstset{language=bash}
\begin{lstlisting}
   git clone https://github.com/MALBECC/lio.git .
\end{lstlisting}




\subsection{Pre-requisites}

\subsection{Compilation}

\subsection{MD-Engine Interfacing}



\section{Tips and tricks - Optimizing your runs}


%%%%%%%%%%%%%%%%%%%%%%%%%%%%%%%%%%%%%%%%%%%%%%%%%%%%%%%%%%%%
