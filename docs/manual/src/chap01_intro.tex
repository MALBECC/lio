%%%%%%%%%%%%%%%%%%%%%%%%%%%%%%%%%%%%%%%%%%%%%%%%%%%%%%%%%%%%
\chapter{Introduction}
%%%%%%%%%%%%%%%%%%%%%%%%%%%%%%%%%%%%%%%%%%%%%%%%%%%%%%%%%%%%
\section{What is LIO?}

Welcome to the LIO project! LIO is a library that can
perform electronic structure calculations using density
functional theory. 


\section{Instalation}

If you are reading this manual, you probably already have a version of lio ready to compile. If you don't, or if
you want to make sure you have the most up-to-date version of the code, all you need to do is either download it
from the git repository online or use git to clone a copy.

For the first option, go to
\textbf{\textit{\href
{https://github.com/MALBECC/lio}
{https://github.com/MALBECC/lio}
}}
and click on the green button that says 
\textit{clone or download}
and click on \textit{Download ZIP}.

For the second one, you can directly run the following
command:

\lstset{language=bash, keywordstyle=\color{violet}, 
morekeywords={clone}}
\begin{lstlisting}
   git clone https://github.com/MALBECC/lio.git .
\end{lstlisting}


\subsection{Pre-requisites}
In addition to an UNIX-like OS, fortran and c++ compilers, LIO depends on LAPACK 
\textit{\href{http://www.netlib.org/lapack/}{LAPACK}} and BLAS libraries for 
linear algebra calculations. In addition, 
\textit{\href{https://developer.nvidia.com/cuda-downloads}{CUDA 6.5 or higher}}
is required for GPU calculations,which unleash LIO's true potential. As of the writing of
this manual, LIO has not yet been tested with CUDA 9.0.

In addition, \textit{\href{https://tddft.org/programs/libxc/}{libxc}} is
required for its usage, although said library is CPU-only. This is entirely optional
as LIO can run without libxc, using only the PBE functional.

\subsection{Compilation}

By default, LIO compiles with GPU options enabled. It is highly recommended to specify
the GPU architecture as a compilation option, since the compiler performs additional
enhancements. After compilation, LIOHOME environment variable should be set to the
current LIO installation directory:

\lstset{language=bash, keywordstyle=\color{violet}, 
morekeywords={export}}
\begin{lstlisting}
   export LIOHOME=/dir/to/lio/
\end{lstlisting}

For a CPU-only compilation, use:

\lstset{language=bash, keywordstyle=\color{violet}, 
morekeywords={make}}
\begin{lstlisting}
   make cuda=0
\end{lstlisting}

If INTEL compilers are present, they can be used by setting the \textit{intel}
option to 1, or to 2 if INTEL MKL usage is also desired. 

\lstset{language=bash, keywordstyle=\color{violet}, 
morekeywords={make}}
\begin{lstlisting}
   make intel=2
\end{lstlisting}

The following is a list of available compilation and their meanings. They can be
used in any combination possible (including several GPU architectures for greater
compatibility). For example, the default LIO compilation could be written as:

\lstset{language=bash, keywordstyle=\color{violet}, 
morekeywords={make}}
\begin{lstlisting}
   make cuda=2 sm30=1 sm52=1 sm61=1
\end{lstlisting}

%%%%%%%%%%%%%%%%%%%%%%%%%%%%%%%%%%%%%%%%%%%%%%%%%%%%%%%%%%%%
\begin{Spacing}{1.0}
\begin{longtable}{ p{.25\textwidth} p{.70\textwidth} }

   \toprule
   \textbf{Variable} & Description \\*
   \midrule \\*
   \endhead

   \bottomrule
   \caption{Compilation options}
   \endfoot

   \textbf{cuda}
   &  \textit{default cuda=2}
   \\*\textit{0, 1, 2}
   & Sets the level of GPU dependencies. cuda=0 means a 
   CPU-only compilation, =1 enables GPU, and =2 includes
   CUBLAS usage for linear algebra operations. \\* \\

   \textbf{intel}
   &  \textit{default intel=0}
   \\*\textit{0, 1, 2}
   & Sets the usage of INTEL compilers. intel=0 means only
   GNU compilers, =1 means INTEL, and =2 uses INTEL MKL 
   instead of BLAS/LAPACK routines. \\* \\

   \textbf{precision}
   &  \textit{default precision=0}
   \\*\textit{0, 1}
   & Sets level of precision in XC calculations. By default
   LIO uses mixed precision; precision=1 sets everything in
   double precision. This is specially useful when attempting
   high-precision geometry optimizations.\\* \\

   \textbf{analytics}
   &  \textit{default analytics=0}
   \\*\textit{0, 1, 2, 3}
   & Setting analytics = 1 specifies a profiling compilation,
   while analytics = 2 and 3 set higher levels of debugging
   information (for usage with gdb, for example).
   \\* \\

   \textbf{smXX}
   &  \textit{default sm30=1 sm52=1 sm61=1}
   \\*\textit{0, 1}
   & Sets a specific GPU architecture for compilation. 
   Available options are sm30, sm35 (Kepler, CUDA $\geq$ 
   5.0), sm50, sm52 (Maxwell and GeForce 980, CUDA $\geq$
   6.5), sm60 and sm61 (Pascal and GeForce 1080, CUDA $\geq$
   8.0).
   \\* \\

\end{longtable}
\end{Spacing}
%%%%%%%%%%%%%%%%%%%%%%%%%%%%%%%%%%%%%%%%%%%%%%%%%%%%%%%%%%%%


\subsection{MD-Engine Interfacing}
LIO can be linked with  \textit{\href{http://ambermd.org/index.php}{AMBER}},
our own \textit{\href{https://github.com/MALBECC/gromacs}{GROMACS}} fork, 
our own \textit{\href{https://github.com/MALBECC/hybrid}{HYBRID}} code for 
QM/MM calculations.

In all three cases, the LIOHOME environment variable should be set to the current
LIO installation directory. In addition to this section, please refer to each
of the software packages' installation manual for further clarifications.

In order to compile AMBER with LIO, AMBER should be compiled after LIO with the
following options set:

\lstset{language=bash, keywordstyle=\color{violet}, 
morekeywords={make, export}}
\begin{lstlisting}
   export AMBERHOME=/dir/to/amber/
   ./configure -lio -noX11 -netcdfstatic gnu
   make clean
   make install
\end{lstlisting}

For a GROMACS-LIO compilation, GROMACS should be compiled after LIO with the 
following options:

\lstset{language=bash, breaklines=true, breakatwhitespace=true,
showstringspaces=false, stringstyle=\color{olive},
keywordstyle=\color{violet}, morekeywords={cmake, make}}
\begin{lstlisting}
   cd gromacs_compilation_directory/
   cmake gromacs_src_dir/ -DGMX_QMMM=1 -DGMX_QMMM_PROGRAM="lio" -DLIO_LINK_FLAGS="-L/usr/lib -L/usr/lib64 -L$LIOHOME/g2g -L$LIOHOME/lioamber -lg2g -llio-g2g" -DGMX_GPU=0 -DGMX_THREAD_MPI=0
   make
   make install
\end{lstlisting}

\section{Tips and tricks - Optimizing your runs}
There are several ways to optimize your production runs, and it is highly recommended
to fine tune these settings if the same system or very similar systems are going to
run for extended periods of time. Please see the corresponding sections in this manual
for a more detailed explanation of these variables.

The first set to tune is \textliovar{rmax} and \textliovar{rmaxs}, which are
related to the integral cut-offs used in both Coulomb and fitting set integrals. The
\textit{higher} the value of \textliovar{rmaxs} and the \textit{lower} the value
of \textliovar{rmax}, the calculations will run faster. Keep in mind, however,
that \textliovar{rmax} should \textit{never} be lower that \textliovar{rmaxs},
and that tuning those values for a faster calculation will certainly result in a loss
of precision. Therefore, the idea is to decrease  \textliovar{rmax} and increase
\textliovar{rmaxs} until the difference in energies and forces stops being
negligible.

Next is the set of options available for the GPU library which performs the exchange-
correlation calculations. The option \textliovar{max\_function\_exponent} indicates
the maximum exponent considered for a fuction in a point of the grid; it should be 
tweaked taking into account the aforementioned criterium: the faster the calculation
is performed, the lesser precision is achieved. 
There are also three other options whose optimal values depend on the GPU architecture
available: \textliovar{little\_cube\_size}, \textliovar{min\_points\_per\_cube},
and \textliovar{sphere\_radius}. These can be tweaked for maximum speed without
worrying about the resulting precision.


%%%%%%%%%%%%%%%%%%%%%%%%%%%%%%%%%%%%%%%%%%%%%%%%%%%%%%%%%%%%
