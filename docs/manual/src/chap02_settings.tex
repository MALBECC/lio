%%%%%%%%%%%%%%%%%%%%%%%%%%%%%%%%%%%%%%%%%%%%%%%%%%%%%%%%%%%%
\chapter{General Settings}
%%%%%%%%%%%%%%%%%%%%%%%%%%%%%%%%%%%%%%%%%%%%%%%%%%%%%%%%%%%%

In this chapter we will describe the basic settings for running the code.
These will be necessary or useful for any of the features of the code
that you might be interested in using.

%%%%%%%%%%%%%%%%%%%%%%%%%%%%%%%%%%%%%%%%%%%%%%%%%%%%%%%%%%%%
\section{General System Description}

\begin{Spacing}{1.0}
   \begin{longtable}{ p{.25\textwidth} p{.70\textwidth} }
   
      \toprule
      \textbf{Variable} & Description \\*
      \midrule \\*
      \endhead
   
      \bottomrule
      \caption{General setup}
      \endfoot

       natom  & $ integer = 0 $ \\
       & Sets the number of atoms only in the QM region. This is ignored when
       calling LIO from another MM engine.\\
       \\
       nsol   & $ integer = 0 $ \\
       & Sets the number of "solvent" atoms, meaning atoms in the MM region. 
       This is also ignored when calling LIO externally. \\
       \\
       charge & $ integer = 0 $ \\
       & Sets the total charge for the QM region. \\
       \\
       OPEN   & $ logical = .false. $\\
       & Indicates whether an open-shell calculation should be performed.\\
       \\
       nUnp   & $ integer = 0 $ \\
       & Indicates the number of unpaired electrons for an open-shell calculation, 
       in a closed-shell calculation this variable is ignored. Keep in mind that 
       the number of unpaired electrons is NOT the multiplicity, rather nUnp = Mult+1.\\
       \\

   \end{longtable}
\end{Spacing}


%%%%%%%%%%%%%%%%%%%%%%%%%%%%%%%%%%%%%%%%%%%%%%%%%%%%%%%%%%%%
\section{Common Inputs and Outputs}

\begin{Spacing}{1.0}
   \begin{longtable}{ p{.25\textwidth} p{.70\textwidth} }
   
      \toprule
      \textbf{Variable} & Description \\*
      \midrule \\*
      \endhead
   
      \bottomrule
      \caption{Common inputs and outputs}
      \endfoot

      style  & $ logical = .false. $ \\
      & Activates a more stylish version of the output. \\
      \\
      verbose & $ integer = 0 $ \\
      & Sets verbose level for output. \\
      &     0 = Nothing is printed save for a small LIO start message.\\
      &     1 = only the input namelist and the final convergence results \\
      &         are printed.\\
      &     2 = information for each iteration is added.\\
      &     3 = GPU module information is added.\\
      &     4 = additional miscellaneous information is printed.\\
      \\
      writeXYZ & $ logical = .false. $\\
      & Writes and xyz file containing the QM region, using the filename
      specified in fcoord. This file is written in each of the MD steps 
      when LIO is used with AMBER/GROMACS/HYBRID. \\
      \\
      fCoord & $ character*20 = 'qm.xyz' $\\
      & Output file containing the QM region. \\
      \\
      timers & $ integer = 0 $ \\
      & Sets the timer level for benchmarking and profiling. \\
      &     0 = no timers. \\
      &     1 = only timers in certain subroutines. \\
      &     2 = full timer summary. \\
      \\
      dBug   & $ logical = .false. $\\
      & Checks for NaNs in Fock and Rho matrices.\\
      \\
      gaussian\_convert & $ logical = .false. $ \\
      & Reads an electron density from Gaussian09 as starting guess. 
      This feature is still experimental. \\
      \\
      print\_coeffs   & $ logical = .false. $\\
      & Prints coefficients and energies for each molecular orbital. \\
      \\      
      VCInp & $ logical = .false. $ \\
      & Reads a MO coefficient restart from frestartin.\\
      \\
      restart\_freq & $ integer = 1 $ \\
      & Writes a MO coefficient restart every restart\_freq iterations into 
      frestart.\\
      \\
      frestart & $ char*20 = 'restart.out' $ \\
      & Name of the output density or coefficient restart file. \\
      \\
      frestartin & $ char*20 = 'restart.in' $ \\
      & Name of the input density or coefficient restart file. \\
      \\
      rst\_dens & $ integer = 0 $ \\
      & When rst\_dens >= 1, the restart read by VCInp must be a density
      matrix (not coefficient matrix) restart. Likewise, when rst\_dens = 2,
      the restart written with restart\_freq is a density matrix restart
      (instead of a coefficient matrix restart). \\
      \\
   \end{longtable}
\end{Spacing}


%%%%%%%%%%%%%%%%%%%%%%%%%%%%%%%%%%%%%%%%%%%%%%%%%%%%%%%%%%%%
\section{Properties Calculations}


\begin{Spacing}{1.0}
   \begin{longtable}{ p{.25\textwidth} p{.70\textwidth} }
   
      \toprule
      \textbf{Variable} & Description \\*
      \midrule \\*
      \endhead
   
      \bottomrule
      \caption{Common inputs and outputs}
      \endfoot

      writeForces  & $ logical = .false. $ \\
      & Performs forces calculation and prints the result to a file named
      "forces". \\
      \\
      dipole & $ logical = .false. $ \\
      & Performs dipole calculation and prints the result to a file named
      "dipole". \\
      \\
      mulliken & $ logical = .false. $ \\
      & Performs a Mulliken population analysis and outputs the result to
      a file called "mulliken".\\
      lowdin & $ logical = .false. $ \\
      & Performs a Lowdin population analysis and outputs the result to
      a file called "mulliken".\\
      fukui & $ logical = .false. $ \\
      & Calculates the condensed-to-atoms Fukui function for the system, 
      calculating the spin-polarized version in open-shell systems.\\
   \end{longtable}
\end{Spacing}

%%%%%%%%%%%%%%%%%%%%%%%%%%%%%%%%%%%%%%%%%%%%%%%%%%%%%%%%%%%%
\section{GPU Options}

These options affect the calculations in the GPU module. For the exchange
correlation integrals, the integration grid is separated into cubes and spheres,
with the smaller cubes (less points per cube) being calculated in CPU while the
rest is calculated in GPU. Most of these options should be tweaked for optimal
performance in a given system.

\begin{Spacing}{1.0}
   \begin{longtable}{ p{.35\textwidth} p{.60\textwidth} }
   
      \toprule
      \textbf{Variable} & Description \\*
      \midrule \\*
      \endhead
   
      \bottomrule
      \caption{GPU Module Options}
      \endfoot
   
      \textbf{gpu\_level}
      &  \textit{default = 4}
      \\*\textit{integer}
      & Determines which calculations are performed by the GPU, only
      available when compiled with cuda > 0.\\*
      &     0/1 = only exchange-correlation integrals.\\
      &     2 = adds QM/MM interaction energies and gradients.\\
      &     3 = adds Coulomb energies and gradients.\\
      &     4 = adds nuclear attraction energies and gradients.\\
      &     5 = adds Coulomb basis fitting energies and gradients.\\
      \\
   
      \textbf{max\_function\_exponent}
      &  \textit{default = 10}
      \\*\textit{integer}
      & Ignores functions with $\lvert exponent \rvert > 
      max\_function\_exponent$. This is only for the exchange-correlation
      calculations.\\* \\
   
      \textbf{little\_cube\_size}
      &  \textit{default = 8.0d0}
      \\*\textit{double precision}
      & Small cube-type point group size (in Angstrom).\\* \\
   
      \textbf{min\_points\_per\_cube}
      &  \textit{default = 1}
      \\*\textit{integer}
      & Minimum number of grid points in a cube.\\* \\
   
      \textbf{assign\_all\_functions}
      &  \textit{default = .false. }
      \\*\textit{logical}
      & Calculate \textbf{all} functions (ignores $max\_function\_exponent$).
      This is intended only as a debug option and its usage is not recommended.\\* \\
   
      \textbf{sphere\_radius}
      &  \textit{default = 0.6d0}
      \\*\textit{double precision}
      & Radius of the sphere-type point groups. 0 means there are no sphere-type
      groups, 1 means all points are contained in sphere-type groups.\\* \\
   
      \textbf{remove\_zero\_weights}
      &  \textit{default = .true. }
      \\*\textit{logical}
      & Discard functions for those whose weight is zero
      (.false. option only remains as a debug option).
      \\* \\
   
      \textbf{energy\_all\_iterations}
      &  \textit{default = .false. }
      \\*\textit{logical}
      & Calculate Exc energy in all SCF iterations. Usually, XC energy is only
      calculated in the final step in order to accelerate calculations. \\* \\
   
      \textbf{free\_global\_memory}
      &  \textit{default = 0.0d0}
      \\*\textit{double precision}
      & Fraction of global GPU memory available for the calculation
      (1 means 100\%).\\* \\
   
   
   \end{longtable}
   \end{Spacing}
%%%%%%%%%%%%%%%%%%%%%%%%%%%%%%%%%%%%%%%%%%%%%%%%%%%%%%%%%%%%
