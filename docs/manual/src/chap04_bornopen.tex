\chapter{Ground State Calculations}


\section{Single-point and Born-Oppenheimer Molecular Dynamics}

Single-point calculations consist in finding the ground state density that minimizes the energy
for a given nuclei distribution. Once found, the program can also calculate the force field of
that density for the MD-engine to move the nuclei.

\begin{Spacing}{1.0}
  \begin{longtable}{ p{.25\textwidth} p{.70\textwidth} }
  
      \toprule
      \textbf{Variable} & Description \\*
      \midrule \\*
      \endhead
  
      \bottomrule
      \caption{BO-MD useful settings.}
      \endfoot
        initial\_guess & $ integer = 0 $ \\
        & Selects the method for calculating a starting guess. This is 
        only useful for the first MD step, since after that the starting
        guess is the electron density from the previous step.\\
        \\
        nMax & $ integer = 100 $ \\
        & Maximum number of SCF iteration steps.\\
        \\
        told  & $ double \Hquad precision = 1.0d-6 $ \\
        & Criterium for density matrix convergence. \\
        \\
        etold & $ double \Hquad precision = 1.0d0 $ \\
        & Criterium for energy convergence. \\
        \\
        DIIS  & $ logical = .true. $\\
        & Uses DIIS algorithm for convergence. \\
        \\
        hybrid\_converg & $ logical = .false. $\\
        & Uses the hybrid convergence algorithm. \\
        \\
        good\_cut & $ double \Hquad precision = 1.0d-3 $\\
        & Sets the threshold to start DIIS when activating hybrid convergence. \\
        \\
        VCInp & $ logical = .false. $ \\
        & Reads a MO coefficient restart.\\
        \\
        restart\_freq & $ integer = 1 $ \\
        & Writes a MO coefficient restart every restart\_freq steps.\\
        \\
        frestart & $ char*20 = 'restart.out' $ \\
        & Name of the output restart file. \\
        \\
        frestartin & $ char*20 = 'restart.in' $ \\
        & Name of the input restart file. \\
        \\
        rst\_dens & $ integer = 0 $ \\
        & rst\_dens = 1 reads a density matrix restart, while rst\_dens = 2
        both reads and writes a density matrix restart. \\
        \\
        basis\_set & $ char*20 = 'DZVP' $ \\
        & Name of the basis set used in the calculation.\\
        \\

  \end{longtable}
\end{Spacing}
    

\section{Geometry optimizations}

Geometry optimizations or energy minimization is the process of finding an atomic arrangement in space where the force on each atom is acceptably close to zero.

    \subsection{Implemenation}
    
    LIO has a simple steepest-descent algorithm. The idea is to move the system in the force direction, at a $\lambda$ step value.
    
    \begin{equation}
      \vec{r}^i_{new}=\vec{r}^i + \lambda  \vec{F}^i  
      \label{new_force}
    \end{equation}
    
    Without a linear search algorithm $\lambda$ is obtained as $\frac{steep\_ size}{|\vec{F}_{max}|}$. If the energy decreases with the movement, the step is accepted; but if the energy increases with the step, the steep is rejected and $\lambda$ is reduced. Each accepted move increases step size a 20\% and each rejected move decreases step size a 50\%.
    
    In a linear search algorithm the system scans the energy as function of $\lambda$ and predicts the best value of $\lambda$ to move the system in the gradient direction.
    
    \begin{table}  [h!]
      \begin{center}
      \begin{tabular}{ c c }

    \textbf{No Linear Search algorithm} & \textbf{Linear Search algorithm} \\

    \begin{tikzpicture}[node distance=2cm]
      \node (start) [startstop] {$\vec{r}$, E};
      \node (forces) [element, below of=start ] {Calculate $\vec{F}$};
      \node (fmax) [element, below of=forces ] {Obtain  $|\vec{F}|_{max}$};
      \node (Enew) [element, below of=fmax ] {move and calculate $E_{new}$};
      \node (Edescend) [decision, below of=Enew, yshift=-0.5cm ] {$E_{new} < E$ ?};
      \node (yes2) [yes, below of=Edescend, xshift=-1.5cm ] {yes};
      \node (no2) [no, below of=Edescend, xshift=1.5cm ] {no};
      \node (increase) [element, below of=yes2, xshift=-1.0cm ] {accept $\vec{r}$, increase steep size 20\%};
      \node (decrease) [element, below of=no2, xshift=1.0cm ] {reject $\vec{r}$, decrease steep size 50\%};
      \node (converge) [decision, below of=increase, yshift=-0.5cm ] {converged?};
      \node (yes1) [yes, below of=converge, yshift=-0.5cm ] {yes};
      \node (no1) [no, right of=converge, xshift=5.5cm ] {no};
      \node (finish) [startstop, below of=yes1 ] {End};

      \draw [arrow] (start) -- (forces);
      \draw [arrow] (forces) -- (fmax);
      \draw [arrow] (fmax) -- (Enew);
      \draw [arrow] (Enew) -- (Edescend);
      \draw [arrow] (Edescend) -- (yes2);
      \draw [arrow] (Edescend) -- (no2);
      \draw [arrow] (yes2) -- (increase);
      \draw [arrow] (increase) -- (converge);
      \draw [arrow] (no2) -- (decrease);
      \draw [arrow] (decrease) |- (Enew);
      \draw [arrow] (converge) -- (yes1);
      \draw [arrow] (converge) -- (no1);
      \draw [arrow] (yes1) -- (finish);
      \draw [arrow] (no1) |- (forces);
    \end{tikzpicture}
    
    &

    \begin{tikzpicture}[node distance=2cm]
      \node (start) [startstop] {$\vec{r}$, E};
      \node (forces) [element, below of=start ] {Calculate $\vec{F}$};
      \node (fmax) [element, below of=forces ] {Obtain  $|\vec{F}|_{max}$};
      \node (Elamb) [element, below of=fmax ] {Obtain  E($\lambda$)};
      \node (lamb) [element, below of=Elamb ] {Obtain best $\lambda$};
      \node (newr) [element, below of=lamb ] {move and recalculate E};
      \node (converge) [decision, below of=newr, yshift=-0.5cm ] {converged?};
      \node (yes1) [yes, below of=converge, yshift=-0.5cm ] {yes};
      \node (no1) [no, right of=converge, xshift=0.5cm ] {no};
      \node (finish) [startstop, below of=yes1 ] {End};

      \draw [arrow] (start) -- (forces);
      \draw [arrow] (forces) -- (fmax);
      \draw [arrow] (fmax) -- (Elamb);
      \draw [arrow] (Elamb) -- (lamb);
      \draw [arrow] (lamb) -- (newr);
      \draw [arrow] (newr) -- (converge);
      \draw [arrow] (converge) -- (yes1);
      \draw [arrow] (converge) -- (no1);
      \draw [arrow] (yes1) -- (finish);
      \draw [arrow] (no1) |- (forces);
    \end{tikzpicture}

       \end{tabular}
       \end{center}
      \label{steep-algorithm}
    \end{table}   
    
Best $\lambda$ in lineal search algorithm is obtained by a quadratic function ajusted using minimum energy of the scan and previous and next points.

    \subsection{Using geometry optimizations}
    
    Adding steep=t in LIO input enables geometry optimization (steepest descent, lineal search by default).
    Convergence criteria are set by Force\_cut and Energy\_cut (5E-4 Hartree/bohr and 1E-4 Hartree by Default).
    The number of minimization steeps is set by n\_min\_steps (500 by default) and initial distance steep is set by minimzation\_steep (by default 0.05 bohr)\\
    It is highly advisable to compile LIO in double precision in order to minimise the error in exchange-correlation forces (precision=1).
        Outputs of geometry optimizations are traj.xyz (atoms coordinates in each steepes descent movement) and optimization.out (steep, energy and others). If verbose=true optimization.out includes the energy of each linear search point.
    
    \subsection{Examples}
    
    Examples of geometry optimization are made in lio/test/13\_geom\_optim.
    
\newpage
\section{Restraints}
LIO may add an extra potential term to the Hamiltonian in order to restrain the distance between specified pairs of atoms.

    \subsection{Implemenation}
    The implementation is a simple harmonic potential over a generalized coordinate $r$.

    \begin{equation}
      U=\frac{1}{2} k [r - l_0]^2  
      \label{E_restrain}
    \end{equation}

    $r$ may be defined as a weighted combination of distances between pairs of atoms.

    \begin{equation}
      r=  \sum_{i} \sum_{j>i} w_{ij} |\vec{r_i} - \vec{r_j}|
      \label{gen_coord}
    \end{equation}

    In this formulation the force over an atom l is:

    \begin{equation}
      \vec{F_l}= -k [r - l_0] \sum_{i} \sum_{j>i} w_{ij} \frac{\vec{r_{ij}}}{r_{ij}} \eta_{ijl}     
      \label{rest_force}
    \end{equation}

    Where $\eta_{ijl}$ is defined as:

    \begin{equation*}
      \eta_{ijl} =
       \begin{cases}
          1 & \text{if $l=i$}\\
         -1 & \text{if $l=j$}\\
          0 & \text{in other case}
       \end{cases}
       \label{eta}
    \end{equation*}


    \subsection{Using Restraints}

    The number of pairs of atoms to be added in the restraint potential(s) is defined by setting the variable number\_restr, and a list of distance restrains have to be added to in an additional lio.restrain file. For example:

    \begin{table}  [H]
      \begin{center}
      \begin{tabular}{ l c c c c c}
         $a_i$ & $a_j$ & index &   k  &    $w_{ij}$   &  $l_0$    \\
         1  &  2 &   0   &  0.1 &    1.0   & 7.86   \\
         3  &  4 &   0   &  0.1 &   -1.0   & 7.86   \\
         7  &  9 &   1   &  0.4 &    2.0   & -2.3   \\
         13 &  1 &   1   &  0.4 &    1.0   & -2.3   \\
         14 &  3 &   1   &  0.4 &   -3.0   & -2.3   \\
         14 &  2 &   2   &  0.2 &    1.0   & 0.5    \\
         8  &  5 &   3   &  0.3 &    1.0   & 3.2    \\
       \end{tabular}
       \end{center}
      \label{lio.restrain}
    \end{table}

Columns $a_i$ and $a_j$ contain the atom numbers in the QM system to be restrained, while the index number determines which distances contribute to a same generalized reaction coordinate. The remaining columns are the force constants (k), weights of that distance in the generalized coordinate ($w_{ij}$) and equilibrium positions in atomic units ($l_0$).

    \subsection{Examples}

    \textbf{1)In lio.in:}
    
    number\_restr = 1
    
        \textbf{in lio.restrain:}

    \begin{table}  [H]
      \begin{center}
      \begin{tabular}{ l c c c c c}
         $a_i$ & $a_j$ & index &   k  &    $w_{ij}$   &  $l_0$   \\
         1  &  2 &   0   &  0.1 &    1.0   & 7.86   \\
       \end{tabular}
       \end{center}
      \label{Tex1}
    \end{table}

    \textbf{Potential added to system:}

    \begin{equation}
      U=\frac{1}{2} 0.1 \Big{[} 1.0 |\vec{r_1} - \vec{r_2}| - 7.86\Big{]}^2  
      \label{Ex1}
    \end{equation}


    \textbf{2)In lio.in:}

    number\_restr = 2

    \textbf{in lio.restrain:}

    \begin{table}  [H]
      \begin{center}
      \begin{tabular}{ l c c c c c}
         $a_i$ & $a_j$ & index &   k  &    $w_{ij}$   &  $l_0$    \\
         1  &  2 &   0   &  0.1 &    1.0   & 7.86   \\
         3  &  4 &   0   &  0.1 &   -1.0   & 7.86   \\
       \end{tabular}
       \end{center}
      \label{Tex2}
    \end{table}

    \textbf{Potential added to system:}

    \begin{equation}
      U=\frac{1}{2} 0.1 \Big{[} 1.0 |\vec{r_1} - \vec{r_2}| - 1.0 |\vec{r_3} - \vec{r_4}| - 7.86\Big{]}^2  
      \label{Ex2}
    \end{equation}


    \textbf{3)In lio.in:}

    number\_restr = 4

    \textbf{in lio.restrain:}

    \begin{table}  [H]
      \begin{center}
      \begin{tabular}{ l c c c c c}
         $a_i$ & $a_j$ & index &   k  &    $w_{ij}$   &  $l_0$    \\
         1  &  2 &   0   &  0.1 &    1.0   & 7.86   \\
         3  &  4 &   0   &  0.1 &   -1.0   & 7.86   \\
         1  &  3 &   1   &  0.3 &    3.5   & -2.31   \\
         7  &  8 &   1   &  0.3 &   -2.2   & -2.31   \\
       \end{tabular}
       \end{center}
      \label{Tex3}
    \end{table}

    \textbf{Potential added to system:}

    \begin{equation}
      U=\frac{1}{2} 0.1 \Big{[} 1.0 |\vec{r_1} - \vec{r_2}| - 1.0 |\vec{r_3} - \vec{r_4}| - 7.86\Big{]}^2 + \frac{1}{2} 0.3 \Big{[} 3.5 |\vec{r_1} - \vec{r_3}| - 2.2 |\vec{r_7} - \vec{r_8}| +2.31\Big{]}^2 
      \label{Ex3}
    \end{equation}

