%%%%%%%%%%%%%%%%%%%%%%%%%%%%%%%%%%%%%%%%%%%%%%%%%%%%%%%%%%%%
\begin{Spacing}{1.0}
\begin{longtable}{ p{.25\textwidth} p{.70\textwidth} }

   \toprule
   \textbf{Variable} & Description \\*
   \midrule \\*
   \endhead

   \bottomrule
   \caption{Compilation options}
   \endfoot

   \textbf{cuda}
   &  \textit{default cuda=2}
   \\*\textit{0, 1, 2}
   & Sets the level of GPU dependencies. cuda=0 means a 
   CPU-only compilation, =1 enables GPU, and =2 includes
   CUBLAS usage for linear algebra operations. \\* \\

   \textbf{intel}
   &  \textit{default intel=0}
   \\*\textit{0, 1, 2}
   & Sets the usage of INTEL compilers. intel=0 means only
   GNU compilers, =1 means INTEL, and =2 uses INTEL MKL 
   instead of BLAS/LAPACK routines. \\* \\

   \textbf{precision}
   &  \textit{default precision=0}
   \\*\textit{0, 1}
   & Sets level of precision in XC calculations. By default
   LIO uses mixed precision; precision=1 sets everything in
   double precision. This is specially useful when attempting
   high-precision geometry optimizations.\\* \\

   \textbf{analytics}
   &  \textit{default analytics=0}
   \\*\textit{0, 1, 2, 3, 4}
   & Setting analytics = 1 specifies a profiling compilation,
   while analytics = 2 and 3 set higher levels of debugging
   information (for usage with gdb, for example) and 4 includes additional checks for some integrals.
   \\* \\

   \textbf{smXX}
   &  \textit{default sm30=1 sm52=1 sm61=1}
   \\*\textit{0, 1}
   & Sets a specific GPU architecture for compilation. 
   Available options are sm30, sm35 (Kepler, CUDA $\geq$ 
   5.0), sm50, sm52 (Maxwell and GeForce 980, CUDA $\geq$
   6.5), sm60 and sm61 (Pascal and GeForce 1080, CUDA $\geq$
   8.0).
   \\* \\

\end{longtable}
\end{Spacing}
%%%%%%%%%%%%%%%%%%%%%%%%%%%%%%%%%%%%%%%%%%%%%%%%%%%%%%%%%%%%
